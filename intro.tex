Trip planning, as an application of planning and scheduling,
has seen substantive implementations by researchers and
developers\cite{bast2015route}.
Some of the planning systems are \tit{multi-modal}; that is,
combining distinct transportation modes, the trip planners
compute optimal routes from sources to destinations.
This notion of ``optimality" generally refers to the computed routes
having minimal total time or total fare.

However, in the eyes of a user, it may be more faceted than just
``fastest" or ``cheapest."
For instance, for a college student who specifies that she will
only walk or take public transit in a trip from Palo Alto to
San Francisco, the computed plan is not necessarily the fastest
(e.g., taking a cab could be faster)
or the cheapest (e.g., walking all the way has no fare).
This happens when a user tells the planner her hard constraints,
called \tit{constraints}.  
The planner then needs to either satisfy the 
constraints in the search process or return failure because they are
over-restrictive.
Moreover, the user might want to further customize the planner by
describing soft constraints, called \tit{preferences}.
For example, an agent wants to travel from school to
downtown, and prefers biking to taking the bus.
Thus, a trip with more biking than bus may be considered better for
the agent than the one with more bus than biking.
In this case, the planner will need to accommodate user preferences
whenever possible in the search of optimal solutions.

Representing and reasoning about constraints and preferences are
fundamental to decision making in automated planning and scheduling
in artificial intelligence.
Son and Pontelli presented a declarative preference language, 
$\mathcal{PP}$, to represent ordinal preferences (e.g., basic desires,
atomic preferences and general preferences) between trajectories of 
a planning problem\cite{son2004planning}.
Bienvenu et al. proposed a first-order preference language, $\mathcal{LPP}$, 
extending $\mathcal{PP}$ by allowing the specification of quantified qualitative
preferences\cite{bienvenu2011specifying}.
Modesti and Sciomachen introduced an ad hoc utility function for weighing
the arcs both with their time and fare\cite{modesti1998utility}.
This relationship between time and fare in the setting of transportation
was studied, and economic models capturing the trade-off between the two
was presented\cite{antoniou2007methodology}.

However, relatively limited effort has been devoted to designing and
implementing real-world multi-modal trip planners that captures user
constraints and preferences over the \tit{cost base}, possibly
extended from the user with auxiliary \tit{cost metrics},
such as crime rates and pollution statistics.
One notable work by Nina et al.\cite{nina2016no} introduced
system \tit{Autobahn} for generating scenic routes using
Google Street View images to train a deep neural network
to classify route segments based on their visual features.
Although \tit{Autobahn} computes scenic routes using computer
vision techniques, it does not account for extensibility
and personalizability.

Using a high-performance graph search
engine\cite{zhou2011dynamic}, we designed and implemented a 
multi-modal trip planner that uses pure
graph-search. This allows us to flexibly combine various modes
(e.g., walk, bike, car, public transit, and taxi) and 
declaratively change constraints and preferences. 
The planner allows the user
to upload \tit{new} mapping data over which to express
constraints and preferences.  For instance, a user might upload a map of crime
in the city, and ask the trip planner to avoid areas where crime is frequent.
Then, the planner takes constraints expressed in \tit{linear
temporal logic} to constrain the search space (e.g., never bike after
transit, and never walk through bad neighborhoods).
Finally, the planner uses a weighted sum, called \tit{preferential cost function}, 
over several cost metrics (e.g., time spend biking, fare on public transit, 
and overall crimes walking through) which can be re-weighted based
on different user preferences. 

Our paper is organized as follows.
In the next section, we present what it means for a planner to be extensible
and formally define the method to incorporating new metrics into the planner.
In the following section, we discuss the two aspects of personalization
in trip planning: constraints and preferences, and how they are modeled and
reasoned with in the setting of multi-modal trip planning.
We then move on to describe the system structure of our graph-search based planner,
and show results obtained from our planner in various occasions.
Finally, we conclude by outlining some future research directions.

%\begin{itemize}
%	\item What's the problem: personalized trip planning: find trips that
%	  are optimal in the eyes of a user, who doesn't care about mode or
%	  provider boundaries, and whose preferences may be more faceted than
%	  ``fastest'' or ``cheapest''.
%	\item Why is it hard: the computational complexity of searching
%	  combinations of multiple modes that can be combined freely is high
%	  (PSPACE-complete\cite{}) which is why existing trip planner exploit problem
%	  structure (such as only one mode per trip) and predefined
%	  preferences (optimization metrics)
%	\item How did we solve it: Using a high-performance graph search
%	  engine\cite{}, we implemented a multi-modal trip planner that uses pure
%	  graph-search. This allows us to flexibly combine various modes
%	  (e.g., car, train, cab, walk) and declarative change soft and hard
%	  constraints. The planner can take constraints expressed in linear
%	  temporal logic to constrain the search space (e.g., never bike after
%	  transit, i.e., I don't want to take my bike on the bus or train).
%	  The planner uses a weighted sum over several metrics (e.g., time
%	  overall, time spend on transit, cost) which can be re-weighted based
%	  on different user preferences. Finally the planner allows the user
%	  to upload {\em new} mapping data over which to express hard and/or
%	  soft constraints. For instance, a user might upload a map of crime
%	  in the city, and ask the trip planner to never walk at night through
%	  an area where crime is frequent.
%%	\item How did we evaluate the features and what are the results:
%%	  ceteris paribus experimentation setup on Amazon Mechanical 
%%		Turk (MTurk\footnote{\url{http://mturk.amazon.com}}) and results, 
%%		asking users to pick between two different trips, one being limited by the
%%	  capabilities of existing planners, and one exhibiting features only
%%	  available in our trip planner.
%\end{itemize}
